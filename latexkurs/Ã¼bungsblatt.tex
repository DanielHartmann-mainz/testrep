\documentclass[a4paper,12pt]{article}
\usepackage{fancyhdr}
\usepackage{fancyheadings}
\usepackage[ngerman,german]{babel}
\usepackage{german}
\usepackage[utf8]{inputenc}
%\usepackage[latin1]{inputenc}
\usepackage[active]{srcltx}
\usepackage{algorithm}
\usepackage[noend]{algorithmic}
\usepackage{amsmath}
\usepackage{amssymb}
\usepackage{amsthm}
\usepackage{bbm}
\usepackage{enumerate}
\usepackage{graphicx}
\usepackage{ifthen}
\usepackage{listings}
\usepackage{struktex}
\usepackage{hyperref}
\usepackage{german}
\usepackage{paralist}

%%%%%%%%%%%%%%%%%%%%%%%%%%%%%%%%%%%%%%%%%%%%%%%%%%%%%%
%%%%%%%%%%%%%% EDIT THIS PART %%%%%%%%%%%%%%%%%%%%%%%%
%%%%%%%%%%%%%%%%%%%%%%%%%%%%%%%%%%%%%%%%%%%%%%%%%%%%%%
\newcommand{\Fach}{Technische Informatik}
\newcommand{\Name}{Jan Lehmann, Daniel Hartmann}
\newcommand{\Seminargruppe}{2}
\newcommand{\Matrikelnummer}{1337}
\newcommand{\Semester}{SS 2020}
\newcommand{\Uebungsblatt}{1} %  <-- UPDATE ME
%%%%%%%%%%%%%%%%%%%%%%%%%%%%%%%%%%%%%%%%%%%%%%%%%%%%%%
%%%%%%%%%%%%%%%%%%%%%%%%%%%%%%%%%%%%%%%%%%%%%%%%%%%%%%

\setlength{\parindent}{0em}
\topmargin -1.0cm
\oddsidemargin 0cm
\evensidemargin 0cm
\setlength{\textheight}{9.2in}
\setlength{\textwidth}{6.0in}

%%%%%%%%%%%%%%%
%% Aufgaben-COMMAND
\newcommand{\Aufgabe}[1]{
  {
  \vspace*{0.5cm}
  \textsf{\textbf{Aufgabe #1}}
  \vspace*{0.2cm}
  
  }
}
%%%%%%%%%%%%%%
\hypersetup{
    pdftitle={\Fach{}: Übungsblatt \Uebungsblatt{}},
    pdfauthor={\Name},
    pdfborder={0 0 0}
}

\lstset{ %
language=java,
basicstyle=\footnotesize\tt,
showtabs=false,
tabsize=2,
captionpos=b,
breaklines=true,
extendedchars=true,
showstringspaces=false,
flexiblecolumns=true,
}

\title{Übungsblatt \Uebungsblatt{}}
\author{\Name{}}

\begin{document}
\thispagestyle{fancy}
\lhead{\sf \large \Fach{} \\ \small \Name{} %- \Matrikelnummer{}%
}
\rhead{\sf \Semester{} \\  }%Übungsgruppe \Seminargruppe{}}
\vspace*{0.2cm}
\begin{center}
\LARGE \sf \textbf{Übungsblatt \Uebungsblatt{}}
\end{center}
\vspace*{0.2cm}

%%%%%%%%%%%%%%%%%%%%%%%%%%%%%%%%%%%%%%%%%%%%%%%%%%%%%%
%% Insert your solutions here %%%%%%%%%%%%%%%%%%%%%%%%
%%%%%%%%%%%%%%%%%%%%%%%%%%%%%%%%%%%%%%%%%%%%%%%%%%%%%%

\Aufgabe{1}
\begin{enumerate}[a)]
    \item Formeln lassen sich einfach mit zwei \$-Zeichen den Text integrieren: $\frac{a + b}{c} = 1$ . Für eine Längere Formel, die in einer eigenen Zeile stehen soll können doppelte \$-Zeichen verwendet werden:
$$ \frac{a + b}{c} = 1 \Rightarrow a + b = c $$
    \item Nummerierte Formeln bietet die \emph{equation}-Umgebung:

        Später kann Formel referenziert werden. 
    \item Mehrzeilige Formeln können durch die \emph{align}-Umgebung realisiert werden:
        \begin{align*}
            ggT(15, 12) &= ggT(3, 12) \\
                        &= ggT(3, 9) \\
                        &= ggT(3, 6) \\
                        &= ggT(3, 3) \\
                        &= 3
        \end{align*}
    Eine sehr umfangreiche Hilfe zu Formeln in \LaTeX{} findet sich in \url{http://de.wikipedia.org/wiki/Hilfe:TeX}.
\end{enumerate}

\Aufgabe{2}
Die Beschriftungen orientieren sich an der Vorlesungsfolie.
\begin{enumerate}[a)]
\item Es gelten die folgenden Identitäten nah dem Verschmelzungsgesetz:
	\begin{equation}
     b = b + b\cdot c \text{ und } c = c + c\cdot b
            \label{myequation}
     \end{equation}
$ \overline{(\overline{a} +\overline{b})}+(\overline{a}\cdot c)
 \stackrel{\text{De\ Morgan}}{=}
 (\overline{\overline{a}}\cdot \overline{\overline{b}})+(\overline{a}\cdot c)
 \stackrel{\text{h)}}{=}
 (a\cdot b)+(\overline{a}\cdot c)\\
 \stackrel{\ref{myequation}}{=}
 a\cdot (b+ (b\cdot c)) + \overline{a}\cdot (c + (c\cdot b))
  \stackrel{\text{Distributivgesetz}}{=}
  (a\cdot b)+(a\cdot (b\cdot c)) + (\overline{a}\cdot c)+ (\overline{a}\cdot (c\cdot b))\\
  \stackrel{\text{Kommutativgesetz}}{=}
  (a\cdot b)+(a\cdot (b\cdot c)) + (\overline{a}\cdot c)+ (\overline{a}\cdot (b\cdot c))\\
  \stackrel{\text{Distributivgesetz}}{=}
  (a\cdot b) + ((a+ \overline{a})\cdot(b\cdot c)) + ( \overline{a}\cdot c)
  =(a\cdot b) + (1\cdot(b\cdot c)) + ( \overline{a}\cdot c)\\
  \stackrel{\text{f)}}{=}
  (a\cdot b) + (b\cdot c) + ( \overline{a}\cdot c)
  \stackrel{\text{De\ Morgan}}{=}
  (a\cdot b)+(b \cdot c)+\overline{(a+\overline{c})}\\
   \stackrel{\text{Kommutativgesetz}}{=}
   (a\cdot b)+\overline{(a+\overline{c})}+(b \cdot c) $\hfill $\Box$\\
   
   
   
   
\item
Belegt man die Variablen durch \[a=1,\ b=1,\ c=0\] ergibt sich:\\
 \[\overline{(\overline{a}+b \cdot c)}\cdot \overline{(a\cdot \overline{b} \cdot c )}+ \overline{(\overline{b}+ \overline{c})} =\overline{(\overline{1}+0 \cdot 1)}\cdot \overline{(1\cdot \overline{0} \cdot 1 )}+ \overline{(\overline{0}+ \overline{1})}\]\[=\overline{0+0}\cdot \overline{1 \cdot 1 \cdot 1}+1\cdot 0=1\cdot 0+0=0\neq 1 = 1+ 1\cdot 0 = a+\cdot cb\]
 Die Ausdrücke sind also i.A. nicht äquivalent.







\end{enumerate}
 
 
%%%%%%%%%%%%%%%%%%%%%%%%%%%%%%%%%%%%%%%%%%%%%%%%%%%%%%
%%%%%%%%%%%%%%%%%%%%%%%%%%%%%%%%%%%%%%%%%%%%%%%%%%%%%%
\end{document}
